
\documentclass[%
twoside,                 % oneside: electronic viewing, twoside: printing
final,                   % or draft (marks overfull hboxes, figures with paths)
10pt]{article}

\listfiles               % print all files needed to compile this document

\usepackage{relsize,makeidx,color,setspace,amsmath,amsfonts}
\usepackage[table]{xcolor}
\usepackage{bm,microtype}

\usepackage{graphicx}

\usepackage{fancyvrb} % packages needed for verbatim environments

\usepackage{minted}
\usemintedstyle{trac}

\usepackage[T1]{fontenc}
%\usepackage[latin1]{inputenc}
\usepackage[utf8]{inputenc}

\usepackage{lmodern}         % Latin Modern fonts derived from Computer Modern

% Hyperlinks in PDF:
\definecolor{linkcolor}{rgb}{0,0,0.4}
\usepackage[%
    colorlinks=true,
    linkcolor=linkcolor,
    urlcolor=linkcolor,
    citecolor=black,
    filecolor=black,
    %filecolor=blue,
    pdfmenubar=true,
    pdftoolbar=true,
    bookmarksdepth=3   % Uncomment (and tweak) for PDF bookmarks with more levels than the TOC
            ]{hyperref}
%\hyperbaseurl{}   % hyperlinks are relative to this root

\setcounter{tocdepth}{2}  % number chapter, section, subsection

% Tricks for having figures close to where they are defined:
% 1. define less restrictive rules for where to put figures
\setcounter{topnumber}{2}
\setcounter{bottomnumber}{2}
\setcounter{totalnumber}{4}
\renewcommand{\topfraction}{0.85}
\renewcommand{\bottomfraction}{0.85}
\renewcommand{\textfraction}{0.15}
\renewcommand{\floatpagefraction}{0.7}
% 2. ensure all figures are flushed before next section
\usepackage[section]{placeins}
% 3. enable begin{figure}[H] (often leads to ugly pagebreaks)
%\usepackage{float}\restylefloat{figure}

\usepackage[framemethod=TikZ]{mdframed}

% --- begin definitions of admonition environments ---

% --- end of definitions of admonition environments ---

% prevent orhpans and widows
\clubpenalty = 10000
\widowpenalty = 10000

% --- end of standard preamble for documents ---


% insert custom LaTeX commands...

\raggedbottom
\makeindex

%-------------------- end preamble ----------------------

\begin{document}



% ------------------- main content ----------------------

% Exercises for PHY981


% ----------------- title -------------------------

\thispagestyle{empty}

\begin{center}
{\LARGE\bf
\begin{spacing}{1.25}
Exercises spring 2016 PHY981
\end{spacing}
}
\end{center}

\begin{center} % date
Spring 2015
\end{center}

\vspace{1cm}

\subsection*{Exercise 1, deadline January 22}

% --- begin paragraph admon ---
\paragraph{Masses and binding energies.}
The data on binding energies can be found in the file bedata.dat at the github address of the course, see
\href{{https://github.com/NuclearStructure/PHY981/tree/master/doc/pub/spdata/programs}}{\nolinkurl{https://github.com/NuclearStructure/PHY981/tree/master/doc/pub/spdata/programs}}

\begin{itemize}
  \item Write a small program which reads in the proton and neutron numbers and the binding energies 
\end{itemize}

\noindent
and make a plot of all neutron separation energies for the chain of oxygen (O), calcium (Ca), nickel (Ni), tin (Sn) and lead (Pb) isotopes, that is you need to plot
\[
S_n= BE(N,Z)-BE(N-1,Z).
\]
Comment your results. 
\begin{itemize}
 \item In the same figures, you should also include the liquid drop model results of Eq.~(2.17) of Alex Brown's text, namely
\end{itemize}

\noindent
\[
BE(N,Z)= \alpha_1A-\alpha_2A^{2/3}-\alpha_3\frac{Z^2}{A^{1/3}}-\alpha_4\frac{(N-Z)^2}{A},
\]
with $\alpha_1=15.49$ MeV, $\alpha_2=17.23$ MeV, $\alpha_3=0.697$ MeV and $\alpha_4=22.6$ MeV.
Again, comment your results. 
\begin{itemize}
 \item Make also a plot of the binding energies as function of $A$ using the data in the file on bindingenergies and the above liquid drop model.  Make a figure similar to figure 2.5 of Alex Brown where you set the various parameters $\alpha_i=0$. Comment your results. 

 \item Use the liquid drop model to find the neutron drip lines   for Z values up to 120.
\end{itemize}

\noindent
Analyze then the fluorine isotopes and find, where available the corresponding experimental data, and compare the liquid drop model predicition with experiment. 
Comment your results.

A program example in C++ and the input data file \emph{bedata.dat} can be found found at the github repository for the course, see \href{{https://github.com/NuclearStructure/PHY981/tree/master/doc/pub/spdata/programs}}{\nolinkurl{https://github.com/NuclearStructure/PHY981/tree/master/doc/pub/spdata/programs}}

Deadline for this exercise is {\bf January 22, 5pm}.  You can hand in electronically by just sending me your github link, or just the file. I digest most formats, from scans to ipython notebooks. The choice is yours. 

\end{document}


