\documentclass[11 pt, a4paper]{article}

\usepackage{amsmath}
\usepackage{amssymb}
\usepackage{graphicx}
\usepackage{color}
\usepackage{listings}
\usepackage{hyperref}

\lstset{language=c++}
\lstset{basicstyle=\tiny}
\lstset{backgroundcolor=\color{white}}
\lstset{frame=single}

\title{PHY981 Exercise 2}
\author{John Bower}
\date{February 17 2016}

\begin{document}
\maketitle

\section{Section}
\[
\hat{P}^{+}_p = a^\dag_{p+}a^\dag_{p-},
\]
and
\[
\hat{P}^{-}_p = a_{p-}a_{p+},
\] 
respectively.

The Hamiltonian (with $\xi=1$) we will use can be written as
\[
\hat{H}=\xi\sum_{p\sigma}(p-1)a_{p\sigma}^{\dagger}a_{p\sigma}
-g\sum_{pq}\hat{P}^{+}_p\hat{P}^{-}_q.
\]

\begin{align*}

P^{+}_p &= a^\dag_{p+}a^\dag_{p-} \\

P^{-}_p &= a_{p-}a_{p+} \\


H&=\xi\sum_{p\sigma}(p-1)a_{p\sigma}^{\dagger}a_{p\sigma}
-g\sum_{pq}a^\dag_{p+}a^\dag_{p-}a_{p-}a_{p+} \\




H&=\xi\sum_{p\sigma}(p-1)a_{p\sigma}^{\dagger}a_{p\sigma}
-g\sum_{pq}\hat{P}^{+}_p\hat{P}^{-}_q

\end{align*}

\end{document}
