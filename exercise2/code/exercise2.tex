\documentclass[11 pt, a4paper]{article}

\usepackage{amsmath}
\usepackage{amssymb}
\usepackage{graphicx}
\usepackage{color}
\usepackage{listings}
\usepackage{hyperref}

\lstset{language=c++}
\lstset{basicstyle=\tiny}
\lstset{backgroundcolor=\color{white}}
\lstset{frame=single}

\title{PHY981 Exercise 2}
\author{John Bower}
\date{February 17 2016}

\begin{document}
\maketitle

\begin{itemize}
\item All documents can be found at \url{https://github.com/johnbower2012/PHY981_work/tree/master/Exercise2}.
\item All code is included at the end of this document.
\end{itemize}

\section*{(3)}

We begin with our definition of a totally anti-symmetric function, $\Phi_{\lambda}^{AS}$,
\begin{equation*}
\Phi_{\lambda}^{AS}(x_1,\dots,x_N;\alpha_1,\dots,\alpha_N) = \frac{1}{\sqrt{N!}}\sum\limits_p (-1)^p P\prod\limits_{i=1}^N \Psi_{\alpha_i}(x_i),
\end{equation*}
where we are are summing over all the possible permutations, $P$, of interchange in our column indices, $x_i$. This is defined such that if two particles occupy the same state, if some $\alpha_i=\alpha_j$, our function collapses to zero.

\subsection*{(3) a.}
As an example, with $N=3$,
\begin{equation*}
\Phi_{\lambda}^{AS}(N=3) = \frac{1}{\sqrt{3!}} \sum\limits_p (-1)^p P \Psi_{\alpha_1}(x_1) \Psi_{\alpha_2}(x_2) \Psi_{\alpha_3}(x_3), \\
\end{equation*}
and so 
\begin{align*}
\Phi_{\lambda}(N=3) = \frac{1}{\sqrt{6}} \bigg[& \Psi_{\alpha_1}(x_1)\Psi_{\alpha_2}(x_2)\Psi_{\alpha_3}(x_3) - \Psi_{\alpha_1}(x_1)\Psi_{\alpha_2}(x_3)\Psi_{\alpha_3}(x_2) \\
				 	+& \Psi_{\alpha_1}(x_2)\Psi_{\alpha_2}(x_3)\Psi_{\alpha_3}(x_1) - \Psi_{\alpha_1}(x_2)\Psi_{\alpha_2}(x_1)\Psi_{\alpha_3}(x_3) \\
					+& \Psi_{\alpha_1}(x_3)\Psi_{\alpha_2}(x_1)\Psi_{\alpha_3}(x_2) - \Psi_{\alpha_1}(x_3)\Psi_{\alpha_2}(x_2)\Psi_{\alpha_3}(x_1) \bigg].
\end{align*}
\subsection*{(3) b.}
We will now show that this new notation is still normalized to one.
\begin{align*}
<\Phi_{\lambda}^{AS}|\Phi_{\lambda}^{AS}> &= \int\prod_{i=1}^Ndx_i |\Phi_{\lambda}^{AS}|^2 \\
&= \frac{1}{\sqrt{N!}}\int\prod_{x=i}^Ndx_i \bigg(\sum\limits_{p_{x_j}} (-1)^{p_{x_j}} P_{x_j} \prod\limits_{j=1}^N \Psi_{\alpha_j}^* (x_j)\bigg)\bigg(\sum\limits_{p_{x_k}} (-1)^{p_{x_k}} P_{x_k} \prod\limits_{k=1}^N \Psi_{\alpha_k} (x_k)\bigg) \\
&= \frac{1}{\sqrt{N!}}\sum\limits_{p_{x_j},p_{x_k}} P_{x_j,x_k} \prod_{i,j,k=1}^N\int dx_i \bigg((-1)^{p_1+p_2}\bigg) \bigg(\Psi_{\alpha_j}^*(x_j) \Psi_{\alpha_k}(x_k)\bigg)
\end{align*}	
Recall that $\int d_{x_k} \Psi_{\alpha_i}^* (x_k) \Psi_{\alpha_j} (x_k) = \delta_{\alpha_i,\alpha_j}$, such that any cross terms integrate to zero. Then,
\begin{align*}
<\Phi_{\lambda}^{AS}|\Phi_{\lambda}^{AS}> &= \frac{1}{\sqrt{N!}}\sum\limits_{p_{x_i}} P_{x_i} \prod_{i,j=1}^N\int dx_j \Psi_{\alpha_i}^*(x_i) \Psi_{\alpha_i}(x_i) \\
&= \prod_{i=1}^N\int dx_i \Psi_{\alpha_i}^*(x_i) \Psi_{\alpha_i}(x_i) \\
&= \prod_{i=1}^N \bigg(1\bigg	) \\
&= 1
\end{align*}

\subsection*{(3) c.}
We will now calculate a general one-body operator, $\hat{F} = \sum_i^N \hat{f}(x_i)$, and general two-body operator, $\hat{G} = \sum\limits_{i>j}^N \hat{g}(x_i,x_j)$, where $\hat{g}$ is invariant under particle interchange, in terms of Slater Determinants.
\begin{align*}
<\Phi_{\lambda}^{AS}|\hat{F}&|\Phi_{\lambda}^{AS}> \hspace{0.2 cm} =\\
& \frac{1}{\sqrt{N!}}\int\prod_{x=i}^Ndx_i \bigg(\sum\limits_{p_{x_j}} (-1)^{p_{x_j}} P_{x_j} \prod\limits_{j=1}^N \Psi_{\alpha_j}^* (x_j)\bigg)\sum\limits_{m=1}^N\hat{f}(x_m)\bigg(\sum\limits_{p_{x_k}} (-1)^{p_{x_k}} P_{x_k} \prod\limits_{k=1}^N \Psi_{\alpha_k} (x_k)\bigg)
\end{align*}
Notice that $\hat{f}(x_i)$ is $\alpha_j$ independent, so all cross terms go to zero, as before.
\begin{align*}
&= \frac{1}{\sqrt{N!}}\sum\limits_{p_{x_i}} P_{x_i}\int\prod_{x=i}^Ndx_i \bigg(\Psi_{\alpha_i}^* (x_i)\bigg(\sum\limits_{m=1}^N\hat{f}(x_m)\bigg) \Psi_{\alpha_i} (x_i)\bigg) \\
&= \sum\limits_{m=1}^N\int dx_m \bigg(\Psi_{\alpha_m}^* (x_m)\hat{f}(x_m)\Psi_{\alpha_m} (x_m)\bigg) \\
&= \sum\limits_{m=1}^N <\Psi_{\alpha_m}^*|\hat{f}(x_m)|\Psi_{\alpha_m}>
\end{align*}
Therefore $\hat{F}$ for a two-body SD is
\begin{equation*}
\boxed{<\Phi_{\alpha_1\alpha_2}^{AS}|\hat{F}|\Phi_{\alpha_1\alpha_2}^{AS}> \hspace{0.2 cm} = \hspace{0.2 cm} <\alpha_1|\hat{f}(x_1)|\alpha_1> + <\alpha_2|\hat{f}(x_2)|\alpha_2>}.
\end{equation*}
\newline
\newline
Performing the same for a two-body operator, $\hat{G} = \sum_{i>j}^N\hat{g}(x_i,x_j)$,
\begin{align*}
<\Phi_{\lambda}^{AS}|\hat{G}&|\Phi_{\lambda}^{AS}> \hspace{0.2 cm} = \\
& \frac{1}{\sqrt{N!}}\int\prod_{x=i}^Ndx_i \bigg(\sum\limits_{p_{x_j}} (-1)^{p_{x_j}} P_{x_j} \prod\limits_{j=1}^N \Psi_{\alpha_j}^* (x_j)\bigg)\sum\limits_{m<n}^N\hat{g}(x_m,x_n)\bigg(\sum\limits_{p_{x_k}} (-1)^{p_{x_k}} P_{x_k} \prod\limits_{k=1}^N \Psi_{\alpha_k} (x_k)\bigg).
\end{align*}
Note that $\hat{G}$ allows for the mixing of single cross terms, but higher order mixing integrates to zero. Thus we are left with
\begin{align*}
&= \frac{1}{\sqrt{N!}} \sum\limits_{m<n}^N\sum\limits_{i<j}^N \int dx_mdx_n \bigg(\Psi_{\alpha_i}^*(x_m)\Psi_{\alpha_j}^*(x_n)\hat{g}(x_m,x_n)\Psi_{\alpha_i}(x_m)\Psi_{\alpha_j}(x_n) \\
&\hspace{5 cm} -\Psi_{\alpha_i}^*(x_m)\Psi_{\alpha_j}^*(x_n)\hat{g}(x_m,x_n)\Psi_{\alpha_i}(x_n)\Psi_{\alpha_j}(x_m)\bigg) \\
&= \sum\limits_{m<n}^N \int dx_mdx_n \bigg(\Psi_{\alpha_m}^*(x_m)\Psi_{\alpha_n}^*(x_n)\hat{g}(x_m,x_n)\Psi_{\alpha_m}(x_m)\Psi_{\alpha_n}(x_n) \\
& \hspace{5 cm} - \Psi_{\alpha_m}^*(x_m)\Psi_{\alpha_n}^*(x_n)\hat{g}(x_m,x_n)\Psi_{\alpha_m}(x_n)\Psi_{\alpha_n}(x_m)\bigg) \\
&= \sum\limits_{m<n}^N \bigg(<\Psi_{\alpha_m}^*\Psi_{\alpha_n}^*|\hat{g}(x_m,x_n)|\Psi_{\alpha_m}\Psi_{\alpha_n}> - <\Psi_{\alpha_m}^*\Psi_{\alpha_n}^*|\hat{f}(x_m)|\Psi_{\alpha_n}\Psi_{\alpha_m}>\bigg)
\end{align*}
And so $\hat{G}$ for a two-body operator is
\begin{equation*}
\boxed{<\Phi_{\alpha_1\alpha_2}^{AS}|\hat{G}|\Phi_{\alpha_1\alpha_2}^{AS}> \hspace{0.2 cm}=\hspace{0.2 cm} <\alpha_1\alpha_2|\hat{g}(x_1,x_2)|\alpha_1\alpha_2> - <\alpha_1\alpha_2|\hat{g}(x_1,x_2)|\alpha_2\alpha_1>}.
\end{equation*}

It should be noted that the determinant notation is merely a convenient representation for expressing an otherwise cumbersome function. The goal of the summation over permutations is to provide an entirely anti-symmetric function such that it represents the reality of fermions, that is that if any two particles occupy the same state the function collapses to zero. The shorthand of representating this as $|\alpha_1\dots\alpha_n>$ simply has the labels for position understood, such that position one allots $x_1$, position two $x_2$, and so forth to position $n$ alloting $x_n$. As such, if one were to switch the ordering of the $\alpha_i$s as $|\alpha_2\alpha_1\alpha_3\dots\alpha_n>$, one would have $\alpha_2\rightarrow x_1, \alpha_1\rightarrow x_2, \dots, \alpha_i\rightarrow x_i$.

In addition a permutation symmetry (labeling of particles should not matter), operators $\hat{F}$ and $\hat{G}$, it is important to note how many SD interchanges each allowed in its final form. We saw that $\hat{F}$ allowed none, whereas $\hat{G}$ allowed none or one. This trend continues such that an $n$-body operator allows up to $n-1$ mixings. In this way, we see that any one body operator, $\hat{F}$, merely measures the system as individual particles. Any two body operator, $\hat{G}$, considers the interactions of two particles. Similarly, an $n$-body operator considers the interactions of $n$-particles.

\section*{(4)} 
\end{document}
